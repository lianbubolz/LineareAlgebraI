\documentclass[12pt, letterpaper]{article}
\usepackage{geometry}
 \geometry{
 a4paper,
 total={170mm,257mm},
 left=20mm,
 top=20mm,
 }
\usepackage{amssymb}
\usepackage{amsthm}
\usepackage{amsmath}

\title{Lineare Algebra Übungszettel 1}
\author{Moritz Brand, Lian Bubolz tzttt}
\date{}

% This line here is a comment. It will not be typeset in the document.

\begin{document}
\maketitle
\paragraph{Aufgabe H1.}

Seien $A, B, X, Y$ Mengen, und sei $f : X \to  Y$ eine Abbildung. 
Geben Sie für jede der folgenden Behauptungen einen Beweis oder ein Gegenbeispiel an:

\begin{enumerate}

    \item Für alle Teilmengen $Y_1$ und $Y_2$ von $Y$ gilt: 
        $f^{-1}(Y_1 \cup Y_2) = f^{-1}(Y_1) \cup f^{-1}(Y_2)$.

    \item Für alle Teilmengen $X_1$ und $X_2$ von $X$ gilt: 
        $f(X_1 \cap X_2) = f(X_1) \cap  f(X_2)$.

    \item Für alle $X'\subseteq X$ und $Y'\subseteq Y$ gilt $f(X' \cap f^{-1}(Y')) = f(X') \cap Y'$.
    \item $(A \times X) \cap (B \times Y) = (A \cap B) \times (X \cap Y)$. 
    \item $(A \times X) \cup (B \times Y) = (A \cup B) \times (X \cup Y)$. 

\end{enumerate} 
 
\begin{proof}
    \hspace*{10mm} \par

    \begin{enumerate}

        

        \item Für alle Teilmengen $Y_1$ und $Y_2$ von $Y$ gilt: 
            $f^{-1}(Y_1 \cup Y_2) = f^{-1}(Y_1) \cup f^{-1}(Y_2)$.

        \item Sei $X = \{-1,1\}$ mit $X_1 = \{-1\}, \ X_2=\{1\}$, $Y = \{1\}$ und die Abbildung $f : X \to  Y$ durch $x\mapsto 1$ definiert.
        Dann ist per Definition $X_1 \cap X_2 = \emptyset$ und dadurch auch $f(X_1 \cap X_2) = \emptyset$. 
        
        Doch $f(X_1) \cap  f(X_2) = \{1\}$ wodurch die Behauptung widerlegt wurde.

        \item AHHHHHHHHH
        \item AHHHHHHHHH
        \item AHHHHHHHHH
    \end{enumerate} 

\end{proof} 

\paragraph{Aufgabe H2.}

Sei $X \neq \emptyset $ eine endliche Menge und sei $f : X \rightarrow X$ eine Abbildung.
Zeigen Sie: Es existiert ein $m \geq  1$ mit der Eigenschaft: Es gibt ein $x \in X$ mit
$f^m(x) = x$. ($f^m = f \circ \cdots \circ f$ ist die $m$-fache Komposition von $f$).

\begin{proof}
    Wir teilen den Beweis in keine Surjektivität und Surjektivität auf: \par
    (1) Wenn $f(x)$ nicht surjektiv ist $\exists! a \in X : \forall  n \geq |X|, n \in \mathbb{N}$, $: f^n(x)=a $, 
    da $X$ endlich ist und aufgrund der fehlenden Surjektivität ($\exists q \in X : \forall p \in X : q \neq f(p) \Rightarrow |X|>|f(X)|) $) 
    $\forall  b < (|X|)/(|X|-|f(X)|), b \in \mathbb{N}$ : $|f^{b+1}(X)| < |f^b(X)|$.
    In diesem Fall wählen wir $x=a$  und $m=|X|$. \par
    (2) Ist $f(x)$ surjektiv, wählen wir ein beliebiges, festes $x \in X$, wofür zwangsläufig $x \in F=\{f(x),\ldots,f^{|X|+1}(x)\}$ gilt, 
    da es nur $|X-1|$ Elemente ungleich $x$  in $f(X)=X$ gibt, aber $|X|$ Elemente in der Menge $F$.
    Nummerieren wir die Elemente in $F$ mit $i\in\{0,1,\ldots,n+1\}$, wählen wir $m$ so, dass $m = min(\{n | f^n(x)=x \land n>0 \})$
    und haben so, per Definition, eine m-fache Komposition für die $f^m(x)=x$ gilt.
\end{proof}

\paragraph{Aufgabe H3.}

Sei $K$ ein Körper. Zeigen Sie:
\par
    (i) Zu jedem $a \in K$ gibt es nur ein Element $b \in K$ mit $a + b = 0$;\\ Für alle $a, b, c, d \in K$ gilt:\par
(ii) Falls $ab = 0$, so gilt $a = 0$ oder $b = 0$, d.h. $K$ ist nullteilerfrei; \par
(iii) $a/b + c/d = (ad + bc)/(bd)$ falls $b \neq 0$ und $d \neq 0$; \par
(iv) $(-a)(-b) = ab$; \par
(v) $-(-a) = a$. \\ \\  
(Geben Sie in jedem Schritt an, welches der Axiome $(A1), \ldots , (A4), (M 1), \ldots , (M 4), (D)$
Sie benutzen.)

\begin{proof} Wir nutzen in jedem Beweis Kommutativität (A2, M2)
    \\ \hspace*{10mm} \par
    (i) Annahme: $\exists b,b' \in K : a+b=0=a+b'$ 
        \par
        $\Rightarrow b \overset{\text{(A3)}}{=} b + 0 \overset{\text{(A4)}}{=} b + (a + b') \overset{\text{(A1)}}{=} 
        (b + a) + b' \overset{\text{(A4)}}{=} 0 + b' \overset{\text{(A3)}}{=} b' $
    \\ \par
    (ii) Annahme: $\exists a,b \in K : a \neq 0 \neq b$ $\land$ $ab=0$ 
        \par
        $\Rightarrow 0 \overset{\text{(Lemma 2.3)}}{=} 0((ab)^{-1}) \overset{\text{(Annahme)}}{=} 
        (ab)(ab)^{-1} \overset{\text{(M4)}}{=} 1$ Widerspruch zu (M3) $\Rightarrow a=0 \lor b = 0$
    \\ \par
    (iii) Seien $ b \neq 0$ und $d \neq 0$ \par \hspace*{7mm} 
        Lemma 1.1 $(b^{-1}d^{-1})^{-1} = bd $ \par \hspace*{7mm} \hspace*{7mm} 
        Beweis : $(b^{-1}d^{-1})^{-1} \overset{\text{(M3)}}{=} (b^{-1}d^{-1})^{-1}(1) \overset{\text{(M4)}}{=} (b^{-1}d^{-1})^{-1}((b^{-1}b)(d^{-1}d)) 
                \overset{\text{(M1,M2)}}{=}$ \par \hspace*{33mm}
                $(b^{-1}d^{-1})^{-1}(b^{-1}d^{-1})(bd) \overset{\text{(M4)}}{=} 1(bd) \overset{\text{(M3)}}{=} bd $ \\

        \par \hspace*{7mm} 
        $a/b + c/d \overset{\text{(Schreibweise)}}{=} ab^{-1}+cd^{-1} \overset{\text{(M3)}}{=}
        1(ab^{-1}+cd^{-1}) \overset{\text{(D)}}{=} 1(ab^{-1}) + 1(cd^{-1}) \overset{\text{(M4)}}{=}$ \par \hspace*{7mm} 
        $(dd^{-1})(ab^{-1}) + (bb^{-1})(cd^{-1}) \overset{\text{(M1,M2)}}{=} a(b^{-1}d)d^{-1} + c(d^{-1}b)b^{-1}
        \overset{\text{(M1,M2)}}{=} ad(b^{-1}d^{-1}) + cb(d^{-1}b^{-1}) \overset{\text{(D)}}{=}$ \par \hspace*{7mm}  
        $b^{-1}d^{-1}((ad)+(cb)) \overset{\text{(Schreibweise)}}{=} (ad + cb)/(b^{-1}d^{-1})^{-1} \overset{\text{(Lemma 1.1)}}{=} (ad + cb)/(bd)$ 
        \\ \par
    (iv) $(-a)(-b) \overset{\text{(A3)}}{=} 0 + (-a)(-b) \overset{\text{(Lemma 2.3)}}{=} 0(-b)+(-a)(-b) \overset{\text{(A4)}}{=}
        (a+(-a))(b)+(-a)(-b) \overset{\text{(D)}}{=}$ \par \hspace*{7mm}   $ab +(-a)b+(-a)(-b) \overset{\text{(D)}}{=} 
        ab+(-a)(b+(-b)) \overset{\text{(A4)}}{=} ab+(-a)0 \overset{\text{(Lemma 2.3, A3)}}{=} ab$ \\
    \par
    (v) $-(-a) \overset{\text{(A3)}}{=} -(-a) + 0 \overset{\text{(A4)}}{=} -(-a) + ((-a)-a+) 
        \overset{\text{(A1)}}{=} (-(-a)+(-a))+a \overset{\text{(A4)}}{=} 0 + a \overset{\text{(A3)}}{=} a$
    \par
\end{proof}


\paragraph{Aufgabe H4.}

Sei $K$ ein Körper, und sei $K^\times  := K \setminus  \{0\}$. Zeigen Sie: Es gibt keine
bijektive Abbildung $e : K \to K^\times$ mit $e(a + b) = e(a)e(b)$ für alle $a, b \in  K$. Sie
dürfen verwenden, dass das Polynom $X^2 - 1$ in $K$ nur die Nullstellen $1$ und $-1$ hat.
Den Fall $char(K) = 2$  (d.h. $1 + 1 = 0$) sollte man getrennt betrachten.

\begin{proof} 
    Wir zeigen per Widerspruchsbeweis, dass es keine bijektive Abbildung geben kann:
    \\ \\
    Annahme: Es gibt eine bijektive Abbildung.
    \\ \par
    $\exists x \in K : e(x) = (-1) \Rightarrow e(x+x)=(-1)(-1) \overset{\text{(H3)}}{=} 1_{K^\times} $
    \par
    Sei $y \in K$, $y\neq 0_K$, $e(0_K + y) = e(0_K)e(y) \Leftrightarrow  e(y) = e(0_K)e(y) \Rightarrow e(0_K) = 1_{K^\times} $
    \par
    $\Rightarrow e(x+x) = e(0_K) \overset{\text{($e^{-1}$)}}{\Rightarrow} x+x = 0_K $
    \par
    Wenn $x+x = 0_K$, dann gilt $0_K=(x+x)x^{-1} = 1_K+1_K \Rightarrow char(K)=2 $
    \par
    $\Rightarrow 1_{K^\times} = e(0_K) = e(1_K+1_K) = e(1_K)e(1_K) = e(1_K)^2 $ \\
    \par
    Wir wissen, dass das Polynom $X^2 - 1$ nur die Nullstellen $1$ und $-1$ hat. 
    \par
    $\Rightarrow e(1_K) = 1_K = e(0_K) \overset{\text{($e^{-1})$}}{\Rightarrow} 1_K = 0_K $ was ein Widerspruch zu den Körperaxiomen ist. \\ 
\end{proof}

\par
\textit{Es ist Funktionen-Party. tan(y), log(x) etc. sind gut drauf, tanzen und lachen. Nur $e^x$ steht einsam in der Ecke. log(x): "Was ist denn los?" $e^x$: "Ich versuche mich ja zu integrieren, aber es kommt immer dasselbe dabei raus…"}





\end{document}